%!TEX root=ndma.tex
\newcommand{\eq}[1]{=_{#1}}

Given two \hdmas{} $\autom_i = (Q_i,\weight{-}_i,q_0^i,\rho_0^i,\trarrow_i,\acc_i)$, $i=1,2$, we now define their \emph{synchronized product} $\autom_1 \syncp \autom_2$. We assume $\weight{q_1}_1 \cap \weight{q_2}_2 = \varnothing$, for all $q_i \in Q_i$.\todo{Dare lemma in cui si dice che posso sempre rinominare i registri?} Some notation: given two disjoint sets of names $S_1,S_2 \subseteq \names$,we write $Eq(S_1,S_2)$ for the set of all equivalence relations $R \subseteq (S_1 \cup S_2) \times (S_1 \cup S_2)$ induced by equations of the form $x_1 \eq{R} x_2$, for $x_i \in S_i$. 

\begin{definition}[Synchronized product of \hdmas]
\label{def:syncp}
 $\autom_1 \syncp \autom_2$ is an automaton $(Q,\weight{-},q_0,\rho_0,\trarrow_i,\acc)$ defined as follows
\begin{itemize}
	\item $Q := \bigcup_{q_1 \in Q_1,q_2 \in Q_2} \{(q_1,q_2)\} \times Eq(\weight{q_1}_1,\weight{q_2}_2)$;
	%
	\item $\weight{(q_1,q_2,R)} := (\weight{q_1}_1 \cup \weight{q_2}_2)_{/R}$, for $(q_1,q_2,R) \in Q$;
	%
	\item $q_0 := (q_0^1,q_0^2,R_0)$, where $R_0$ is given by
	\[
		\inferrule
		{ \rho_0^1(x_1) = \rho_0^2(x_2) \\ x_1 \in \weight{q_0^1},x_2 \in \weight{q_0^2}}
		{ x_1 \eq{R_0} x_2 }
	\]
	\item $\rho_0([x]_{R_0}) = \rho_0^i (x)$ whenever $x \in \weight{q_0^i}_i$, $i \in \{1,2\}$; 
	%(well-defined by definition of $R_0$);
	%
	\item $(q_1,q_2,R) \htr{l}{\sigma} (q_1',q_2',R')$ if and only if there are $q_i \htrind{l_i}{\sigma_i}{i} q_i'$, $i=1,2$, such that $\sigma([x]_{R'}) = [\sigma_i(x)]_R$, for all $x \in \weight{q_i'}$ with $\sigma_i^{-1}(x) \neq \star$, 
	%for $i \in \{1,2\}$ $i=1,2$ 
	%$q_2 \htrind{l_2}{\sigma_2}{2} q_2'$, 
	and one of the following holds: 
	\begin{itemize}
		%	
		\item if $l_1,l_2 \in \names$ then $l_1 =_R l_2$, $l = [l_1]_R$, and $R'$ is the smallest equivalence relation generated by
		\[
			\inferrule*[left=Eq]
			{ x_1 \eq{R} x_2 \\ x_1 \in \Im(\sigma_1),x_2 \in \Im(\sigma_2)}
			{ \sigma_1^{-1}(x_1) \eq{R'} \sigma_2^{-1}(x_2) }
		\]
		%
		\item if $l_1 \in \names$ and $l_2 = \star $ then $l = [l_1]_R$, $[l_1]_R = \{l_1\}$ and $R'$ is generated by \textsc{Eq} and
		\[
			\inferrule
			{ l_1 \in \Im(\sigma_1) }
			{ \sigma_1^{-1}(l_1) \eq{R'} \sigma_2^{-1}(\star) }
		\]
		and $\sigma([\sigma_2^{-1}(\star)]_{R'}) = [l_1]_R$.
		\item if $l_1,l_2 = \star$ then $l=\star$ and $R'$ is generated by \textsc{Eq} and
		\[
			\sigma_1^{-1}(\star) \eq{R'} \sigma_2^{-1}(\star)
		\]
		and $\sigma([\sigma_1^{-1}(\star)]_{R'}) = \star$.
	\end{itemize}
	%and $\sigma([x]_{R'}) = [\sigma_i(x)]_R$ whenever $x \in \weight{q_i'}$, for $i \in \{1,2\}$.
%		\[
%			\sigma([x]_{R'}) = 
%			\begin{cases}
%				[\sigma_1(x)]_R & x \in \weight{q_1'} \\
%				[\sigma_2(x)]_R & x \in \weight{q_2'}
%			\end{cases}
%		\]
\end{itemize}
\end{definition}

\begin{proposition}
Let $((q_1,q_2,R),\rho) \tr{a} ((q_1',q_2',R'),\rho')$ be an edge in the configuration graph of $\autom_1 \syncp \autom_2$, and let $\rho_i = \lambda x \in \weight{q_i}.\rho([x]_R)$, for $i=1,2$. Then $(q_i,\rho_i) \trind{a}{i} (q_i',\rho_i')$ is and edge in the configuration graph of $\autom_i$ with $\rho'_i = \lambda x \in \weight{q_i}.\rho'([x]_{R'})$.
%, for $i=1,2$.
%and $(q_2,\rho_2) \trind{a}{2} (q_2',\rho_2')$ are edges in the configuration graphs of $\autom_1$ and $\autom_2$, respectively, with $\rho'_i = \lambda x \in \weight{q_i}.\rho'([x]_{R'})$.
\end{proposition}
\begin{proof}
We proceed by cases on the transition $((q_1,q_2,R),\rho) \htr{l}{\sigma} ((q_1',q_2',R'),\rho')$ in $\autom_1 \syncp \autom_2$ that induces $((q_1,q_2,R),\rho) \tr{a} ((q_1',q_2',R'),\rho')$:
\begin{description}
	\item[Case $l \in \names$:] Then there are two transitions $q_i \htrind{l_i}{\sigma_i}{i} q_i'$, $i=1,2$, according to \cref{def:syncp}.
	% and $q_2 \htrind{l_2}{\sigma_2}{2} q_2'$ according to \cref{def:syncp}. 
	We have two cases:
	\begin{itemize}
	\item $l_1,l_2 \in \names$.
	Since $l_1 \eq{R} l_2$, we have $\rho_i(l_i) = \rho([l_i]_R) = \rho([l]_R) = a$, so these transitions induce two edges $(q_i,\rho_i) \trind{a}{i} (q_i',\rho_i')$.
	% and $(q_2,\rho_2) \trind{a}{2} (q_2',\rho_2')$.
	Given $x \in \weight{q_i'}$, the following chain of equations, all expanding the definition of the involved function, proves the last part of the statement
	\begin{align*}
		\rho'_i(x) &= \rho_i (\sigma_i (x) ) \\
		&= \rho([\sigma_i(x)]_R) \\
		%&& \text{(by definition of $\rho_i$)}\\
		&= \rho(\sigma([x]_{R'})) \\
		%&& \text{(by definition of $\sigma$)}\\
		&= \rho'([x]_{R'}) 
		%&& \text{(by definition of $\rho'$)}
	\end{align*}
	
	\item if $l_1 \in \names,l_2 = \star$ then $\rho_1(l_1) = \rho([l_1]_R) = a$ and $a \notin \Im(\rho_2)$. In fact, %$a \in \Im(\rho_2)$ only 
	if there was $l_2 \in \weight{q_2}$ such that $\rho_2(l_2) = a$, then $\rho([l_2]_R) = a = \rho([l_1]_R)$, by definition of $\rho$, which implies $[l_1]_R = [l_2]_R$, by injectivity of $\rho$, i.e. $\{l_1,l_2\} \in [l_1]_R$, against the hypothesis $[l_1]_R = \{l_1\}$. So we have $(q_i,\rho_i) \trind{a}{i} (q_i',\rho_i')$, $i=1,2$, and the same equations as the previous case hold; we just have to check the value of $\rho_2'$ on $x = \sigma_2^{-1}(\star)$:
	\begin{align*}
		\rho'_2(x) &= (\rho_2 \circ \sigma_2)\sub{a}{\sigma_2^{-1}(\star)}(x) \\
		&= a \\
		&= \rho([l_1]_R) \\
		&= (\rho \circ \sigma)([x]_{R'}) \\
		& = \rho'([x]_{R'})
	\end{align*}	
	All the equations just apply the definitions of the involved functions.
	%so $\rho([l_1]_R) = \rho([l_2]_R)$, by injectivity of $\rho$.
	%, but this means $, because $[l_1]_R = \{l_1\}$
	\end{itemize}

	\item[Case $l = \star$:] 
	Then we have $q_i \htrind{\star}{\sigma_i}{i} q_i'$, and for $i=1,2$% and $x = \sigma_i^{-1}(\star)$
	\begin{align*}
		\rho'_i(\sigma_i^{-1}(\star)) &= (\rho_i \circ \sigma_i)\sub{a}{\sigma_i^{-1}(\star)}(\sigma_i^{-1}(\star)) \\
		&= a \\
		&= (\rho \circ \sigma) \sub{a}{\sigma^{-1}(\star)}(\sigma^{-1}(\star)) \\
		&= (\rho \circ \sigma) \sub{a}{[\sigma_1^{-1}(\star)]_{R'}}([\sigma_1^{-1}(\star)]_{R'}) \\
		%&& \text{(by $[\sigma_1^{-1}(\star)]_{R'} = [\sigma_i(\star)^{-1}]_{R'}$)} \\
		& = (\rho \circ \sigma) \sub{a}{[\sigma_i^{-1}(\star)]_{R'}}([\sigma_i^{-1}(\star)]_{R'}) && \text{(definition of $R'$)} \\
		&= \rho'([\sigma_i(\star)^{-1}]_{R'})
	\end{align*}
	
	%Then there are $q_i \htrind{l_i}{\sigma_i}{i} q_i'$, $i=1,2$. We have two cases:
%	\begin{itemize}
%		\item 
%	\end{itemize}
\end{description} 
\end{proof}

\begin{corollary}
Given a path $((q_0^1,q_0^2,R_0),\rho_0) \tr{a_0} \dots \tr{a_{n-1}} ((q_n^1,q_n^2,R_n),\rho_n)$ in the configuration graph of $\autom_1 \syncp \autom_2$, there is a path $(q_0^i,\rho_0^i) \tr{a_0} \dots \tr{a_{n-1}} (q_n^i,\rho_n^i)$ in the configuration graph of $\autom_i$ with $\rho_n^i = \lambda x \in \weight{q_n^i}.\rho_n([x]_{R_n})$, for $i=1,2$.
\end{corollary}

\begin{proposition}
$\syncp$ is associative.
\end{proposition}
