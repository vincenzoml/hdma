%!TEX root=ndma.tex

% Explain what kind of languages do we get (refer to the example in the paper)

%\todo{Dire che la prova UP non è banale. Dire che finite representation allows for a very simple proof of decidability of emptiness}

The behavior of processes has traditionally been represented using models with
%represented using models that feature
%forms of 
transition structures, such as labelled transition systems or automata. 
On the other hand, there is also a connection between finite automata and logic formalisms interpreted over finite and infinite words, dating back to \cite{Buchi60,Elgot61}. The possibility of translating modal logic formulas to automata led to model-checking (see e.g.\ \cite{ClarkeS01}).
%On the other hand, automata are also used to characterize subsets of a set of models (e.g., finite or infinite words). In this sense, automata can be considered logic formulas. This is made clearer when observing that several modal logics enjoy a translation to automata. 
%
Processes that feature resource allocation (e.g.\ \cite{MilnerPW92}), in the form of \emph{name allocation}, pose specific challenges. For instance, they have ad-hoc notions of bisimulation, which cannot be captured by standard set-theoretic models.
%Process algebras such as the $\pi$-calculus \cite{MilnerPW92}, where \emph{names} represent resources, and \emph{name allocation} represent resource allocation, pose more challenges.
%Processes that feature resource allocation, such as 
%the behavior of $\pi$-calculus agents \cite{MilnerPW92}, 
%present more problems, exemplified by the definition of bisimilarity in the $\pi$-calculus itself. 
Transition structures that correctly model name allocation have been proposed in various forms, including coalgebras over presheaf categories \cite{FioreT01,BonchiBCG11,Miculan08,GhaniYV04,SammartinoM14}, history-dependent automata \cite{MontanariP05}, that are coalgebras over named sets \cite{CianciaM10}, and automata over nominal sets \cite{BojanczykKL11}. An equivalence between their underlying categories has been established in \cite{GadducciMM06,CianciaKM10}. More recently, the field of \emph{nominal automata} has essentially used the same structures, no longer as semantic models, but rather as acceptor of languages of finite words (see e.g., \cite{Tze11,KST12,GC11,BojanczykKL11}). In particular, the obtained languages are based on infinite alphabets, but still feature some form of finite memory (in fact, the well known \emph{register automata} of Francez and Kaminski can be regarded as nominal automata, see \cite{BojanczykKL11}). 


The case of infinite words over nominal alphabets is more problematic, as an infinite word over an infinite alphabet is generally not finitely supported. Consider a machine that reads any symbol from an infinite, countable alphabet, and never stores it. Clearly, such a machine has finite (empty) memory. The set of its traces is simply described as the set of all infinite words over the alphabet. However, in the language we have various species of words. Some of them are finitely supported, e.g.\ words that consist of the infinite repetition of a finite word. Some others are not finitely supported, such as the word enumerating all the symbols of the alphabet. Such words lay inherently out of the realm of nominal sets. However, the existence of these words does not give to the language infinite memory. More precisely,  words without finite support can not be ``singled out'' by a finite memory machine; if a machine accepts one of those, then it will accept infinitely many others, including finitely supported words.  

The aim of this work is to translate the intuitions in the previous paragraphs into precise mathematical terms, in order to define a class of languages of infinite words over infinite alphabets, enjoying finite-memory properties. We extend automata over nominal sets to handle infinite words, by imposing a (Muller-style) acceptance condition 
over the \emph{orbits} (not the states!) of automata. By doing so, it turns out that our languages not only are finite-memory, but they retain computational properties, such as closure under boolean operations and decidability of emptiness (thus, containment and equivalence), which we prove by providing finite representations, and effective constructions. 

\begin{itemize}
	\item Systems with infinite behavior: peer to peer, cloud...
	\item Resource allocation is a key operation. 
	\item Typically modeled via nominal calculi, where resources are regarded as names and resource creation as name generation. They admit a meta-model based on presheaves, but ``more concrete'' finite-state operational models are needed, in order to do model-checking and verification.
	\item We introduce deterministic automata on infinite words and we prove that they are closed.. and .. are decidable. This allows model-checking to be carried out for such systems.
	\item {\bf Va tolta tutta la parte sulla UP determinacy: non è il risultato principale}
\end{itemize}

Languages of infinite words are of paramount importance in logics and computer science. Their usage scenarios range from decidability proofs in logics, 
 to applications of relevant practical impact, such as model checking and learning of logical properties. Just as in the case of finite words, these languages are typically defined on finite alphabets. However, there are cases in which the alphabet is infinite, e.g.\
%; think e.g., about 
\emph{data words} (see \cite{Seg06} for  a survey), or \emph{nominal calculi} \cite{MPW92}. Languages of \emph{finite} words over infinite alphabets have thoroughly been studied in the literature (see e.g., \cite{KF94,Tze11}). 
It is nowadays clear that register automata, and languages of infinite alphabets, are often expressible as automata over \emph{nominal sets} \cite{GP02}, which are in turn equivalent  to history-dependent automata \cite{Pistore99,FioreS06,GadducciMM06}. 

Several recent papers (see e.g., \cite{Tze11,KST12,GC11}) deal with nominal automata. The paper \cite{BojanczykKL11} discusses languages that are expressible using generalised notions of nominal sets. The same point of view led to the developments described in \cite{BBKL12,LP13,BKLT13}. All these results may be identified as parts of an emergent
\emph{nominal computation theory}. Nominal sets introduce the key notion of \emph{finite support}, that can be regarded as a finite memory property. From the automata-theoretic perspective, languages of finite words over infinite (nominal) alphabets are treated in a satisfactory way by resorting to an \emph{orbit-finite} set of states, equipped with an \emph{equivariant} transition relation, and equivariant acceptance condition. Finite words have finite support, thus the set of all words forms a nominal set. 

The case of infinite words over nominal alphabets is more problematic, as an infinite word over an infinite alphabet is generally not finitely supported. Consider a machine that reads any symbol from an infinite, countable alphabet, and never stores it. Clearly, such a machine has finite (empty) memory. The set of its traces is simply described as the set of all infinite words over the alphabet. However, in the language we have various species of words. Some of them are finitely supported, e.g.\ words that consist of the infinite repetition of a finite word. Some others are not finitely supported, such as the word enumerating all the symbols of the alphabet. Such words lay inherently out of the realm of nominal sets. However, the existence of these words does not give to the language infinite memory. More precisely,  words without finite support can not be ``singled out'' by a finite memory machine; if a machine accepts one of those, then it will accept infinitely many others, including finitely supported words.  

The aim of this work is to translate the intuitions in the previous paragraphs into precise mathematical terms, in order to define a class of languages of infinite words over infinite alphabets, enjoying finite-memory properties. We extend automata over nominal sets to handle infinite words, by imposing a (Muller-style) acceptance condition 
over the \emph{orbits} (not the states!) of automata. By doing so, it turns out that our languages not only are finite-memory, but they retain computational properties, such as closure under boolean operations and decidability of emptiness (thus, containment and equivalence), which we prove by providing finite representations, and effective constructions. 
%Moreover, we prove that the obtained languages are determined by their \emph{ultimately periodic} fragments, just as in the classical result by B\"uchi \cite{Buchi62}. %\cite{CalbrixNP93}. 
%This clarifies the intuition about accepting ``infinitely many other words'' for machines accepting a word which is not finitely supported. 
%
%The proof itself is non-trivial, as one has to deal with freshness and finite memory, and it crucially depends on the usage of finite representations.
%Being determined by ultimately periodic fragments is
%a relevant property for classical automata, whose consequences have probably not yet been explored in full.
%
%For example, such property has been used in learning of languages of infinite words \cite{FCCTW08}, or to find canonical representatives up-to language equivalence, in a coalgebraic flavour \cite{CV12}. We expect that further exploitation of ultimately-periodic fragments may also be beneficial for 
%our automata.%



