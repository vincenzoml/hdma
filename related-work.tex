%!TEX root=ndma.tex
\paragraph{Related work.}	
Automata over infinite data words have been introduced to prove decidability of satisfiability for many kinds of logic: LTL with freeze quantifier \cite{DemriL09}; safety fragment of LTL \cite{Lazic11}; $FO$ with two variables, successor, and equality and order predicates \cite{BojanczykDMSS11}; EMSO with two variables, successor and equality \cite{KaraST12}; generic EMSO \cite{Bollig11}; EMSO with two variables and LTL with additional operators for data words \cite{KaraT10}. The main result for these papers is decidability of nonemptiness. These automata are ad-hoc, and often have complex acceptance conditions, while we aim to provide a simple and seamless nominal extension of a well-known class of automata. We can also cite variable finite automata (VFA) \cite{GrumbergKS10}, that are automata that recognize patterns specified through ordinary finite automata, with variables on transitions. Their version for infinite words (VBA) relies on B\"uchi automata. VBA are not closed under complementation and determinism is not a syntactic property.
%, nor the existence of deterministic versions is decidable. 
For our, determinism is easily checked and we have closure under complementation.
%, and they are more expressive: there is no VBA for the language of words with distinct consecutive symbols.
%\todo{NOn riconosco il lingussio
% 
% \begin{enumerate}
% 	\item LTL with the Freeze Quantifier and Register Automata \cite{DemriL09}: the paper investigates relative expressiveness and complexity of standard decision problems for LTL with the freeze quantifier ($LTL\downarrow$ ), 2-variable first-order logic ($FO^2$ ) over data words, and register automata. To this purpose, they introduce a spectrum of register automata for infinite words with weak Muller acceptance \cite{MullerSS86}. These automata, although powerful, are very ad-hoc, while our \hdmas{} are seamless extensions of Muller automata to the nominal setting. The deterministic, one-way version of their automata is shown to posses closure under boolean operations\footnote{Theorem2.7} and nonemptiness for them is decidable\footnote{Theorem5.1}.
% 	
% 	\item Variable Automata over Infinite Alphabets (\cite{GrumbergKS10}): they introduce (non-deterministic) variable finite automata (VFA), i.e.\ automata that recognize patterns specified through ordinary finite automata with variables on transitions. Non-deterministic VFAs are closed under all boolean operations except complementation. Deterministic VFAs are also closed under complementation, but determinism cannot be characterized as a restriction on the structure of VFAs. Indeed, not all VFAs have a deterministic counterpart, and the associated decision problem is undecidable. They introduce VBAs, a variant of VFAs recognizing infinite words, based on B\"uchi automata. However, like in the B\"uchi casa, deterministic VBAs are not closed under complementation. Determining whether an automaton like ours is deterministic is a simple check on its structure, and we have closure under complementation. We argue that languages of VBAs are contained in those of \hdmas{}, as B\"uchi automata have the same languages as deterministic Muller automata, and such containment is strict: e.g.\ there is no VFA for the language of words with consecutive distinct symbols.
% 	
% 	\item Safety Alternating Automata on Data Words (\cite{Lazic11}): considers one-way alternating automata with 1 register with the safety acceptance mechanism over infinite data words. The languages of such automata are safety properties: every rejected data $\omega$-word has a finite prefix such that every other data $\omega$-word which extends it is also rejected. They are closed under Boolean operations, non emptiness is undecidable with the weak acceptance mechanism, but decidable with safety acceptance mechanism. They do not mention deterministic versions.
% 	
% 	\item Two-variable logic on data words (\cite{BojanczykDMSS11}): data words are strings of pairs $(l,d)$ of a label $l$ and a data value $d$. The paper introduces data automata, that are pairs of automata acting in sequence. Closed under all boolean operators except complementation. They are more powerful than register automata: they can recognize the language of words with all different symbols. Then they introduce data $\omega$-automata, where one of the automata has a B\"uchi acceptance condition. Emptiness is decidable for such automata. No other properties are proved. Logic considered is $FO$ with two variables, equality and successor/order predicates.
% 	
% 	\item Feasible Automata for Two-Variable Logic with Successor on Data Words (\cite{KaraST12}): they introduce weak data automata. Nonemptiness decidable. Extension to $\omega$-words again via B\"uchi automata. Nonemptiness is typically studied when characterizing logic formulae as automata: if the language is nonempty then the formula is satisfiable. Logic considered is EMSO with two variables.
% 	
% 	\item Extending Büchi Automata with Constraints on Data Values: \cite{KaraT10} introduces B\"uchi automata on data $\omega$-words, whose acceptance conditions relies on contraints. Nonemptiness is decidable. These automata are used to prove that satisfiability for $\exists MSO^2(+1,\sim)$ and $LTL$ with additional operators for data values is decidable on $\omega$-words.
% 	
% 	\item An automaton over data words that captures EMSO logic (\cite{Bollig11}): introduce (non-deterministic) $\omega$-class register automata to deal with existential quantifier that expresses infinite behavior. They capture existential MSO over an arbitrary number of variables. 
% \end{enumerate}
