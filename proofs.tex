%!TEX root=ndma.tex

\begin{proof}[of \cref{thm:languages-finitely-supported}]
% Let $2=\{0,1\}$ with $\pi\cdot 0 = 0$, $\pi\cdot 1 = 1$, thus $\supp(0)=\supp(1)=\emptyset$. Let $h : Q \times \names^\omega \to 2$ be such that $h(q,\alpha) = 1$ if and only if $\alpha \in \Lang_q$. 
By properties of nominal sets, for $x$ finitely supported and $f$ equivariant, $f(x)$ is finitely supported with $\supp(f(x))\subseteq \supp(x)$. Let $h : Q \to \Pow(\names^\omega)$ be the function mapping each $q$ to $\Lang_q$. We need to show that $h$ is equivariant, that is, $h(\pi \cdot q) = \{ \pi \cdot \alpha \mid \alpha \in \Lang_q\}$. Without loss of generality, we shall prove the right-to-left inclusion. Then, since $\pi$ and $q$ are arbitrary, one can prove the left-to-right inclusion starting from the state $\pi \cdot q$ and the permutation $\pi^{-1}$. Let $\alpha \in \Lang_q$. We shall prove that $\pi\cdot\alpha \in \Lang_{\pi\cdot q}$. Consider the unique (accepting) run $r$ of $\alpha$ from $q$, and  the unique run $r'$ of $\pi \cdot \alpha$ from $\pi \cdot q$. By equivariance of the transition function, and definition of run, for all $i$, we have $r'_i = \pi \cdot r_i$, thus $\orb(r'_i)=\orb(r_i)$, therefore $\Inf(r')= \Inf(r)$. 
\qed 
\end{proof}
%
\begin{proof}[of \cref{lem:deterministic-configuration-graph}]
 For each $a$, if $a \in \Im(\rho_1)$, recalling that $\rho_1$ is injective, there is $l \in \weight{q_1}$ with $\rho_1(l) = a$. By definition of \hdma, there is exactly one transition labelled with $l$, let it be $q_1 \htr{l}{\sigma} q_2$. Then by definition of configuration graph, we have $(q_1,\rho_1) \tr a (q_2,\rho_1 \circ \sigma)$. Since $\rho_1$ is injective, there can not be other transitions labelled with $a$ in the configuration graph. If $a \notin \Im(\rho_1)$, consider the only transition with label $\star$ from $q_1$, namely $q_1 \htr{\star}{\sigma} q_2$.  Then we have $(q_1,\rho_1) \tr a (q_2,(\rho_1 \circ \sigma)\sub{a}{\sigma^{-1}(\star)})$ in the configuration graph; this transition is unique by definition.
\end{proof}

\begin{proof}[of \cref{pro:nset-to-nom}]
 A run $\run$ in the configuration graph clearly is also a run in the obtained automaton, and $\inf(\run)$. As $\orb(q,\rho) = \{(q,\rho')\mid \rho' : \weight{q} \inj \omega \}$, also acceptance is the same on both sides. By \cref{lem:deterministic-configuration-graph} we get determinism. The proof is completed by noting that the obtained transition function is equivariant. For this, chose a transition $(q_1,\rho_1) \tr a (q_2,\rho_2)$ in the configuration graph, and look at \cref{def:configuration-graph}, thus consider a corresponding \hdma\ transition $q_1 \htr{l}{\sigma} q_2$. The case when $l \in \weight{q_1}$ is straightforward. When $l=\star$, thus $(q_1,\rho_1) \tr a (q_2,(\rho_1 \circ \sigma)\sub{a}{\sigma^{-1}(\star)})$ consider the permuted configuration $(q_1,\pi\circ\rho_1)$. Since $a \notin \Im(\rho_1)$, also  $\pi(a) \notin \Im (\pi\circ\rho_1)$, thus we have a transition $(q_1,\pi\circ\rho_1)\tr{\pi(a)}(q_2,(\pi \circ \rho_1\circ \sigma)\sub{\pi(a)}{\sigma^{-1}(*)})$, which is precisely the required permuted transition. \qed
\end{proof}

%\begin{proof}[of \cref{prop:equivalence-ndma-hdma}]
 
%\end{proof}
\begin{proof}[of \cref{prop:ndma-to-hdma}]
For $x$ an element of a nominal set, let $o_x$ be a chosen canonical representative of $\orb(x)$, $o_{S \subseteq X} = \{o_x \mid x \in S\}$ and $\sigma_q$ be a chosen permutation such that $\sigma_q \cdot o_q = q$.	Let $(Q,\tr{},q_0,\acc)$ be an nDMA. There is an \hdma{} accepting the same language, namely,  $(o_Q,\weight-,\htr{}{},o_{q_0},\sigma_{q_0},\{ \{ o_x \mid x \in A\} \mid A \in \acc \})$, 
%
where for all $q$, we let $\weight{o_q} = \supp(o_q)$. For each nDMA transition $o_{q_1} \tr a q_2$, if $a \in \supp(o_{q_1})$, then we let $o_{q_1} \htr{a}{\sigma_{q_2}} o_{q_2}$; otherwise, we let $o_{q_1} \htr{\star}{\sigma_{q_2}\sub{*}{a}} o_{q_2}$. It is straightforward to check that the configuration graph of this automaton accepts the same language as the original nDMA. This is already well understood from the equivalence of categories between named sets and nominal sets, see e.g.\ \cite{GadducciMM06}.
\end{proof}


\begin{proof}[of \cref{prop:edges-correspondence}]
Let $C = ((q_1,q_2,R),\rho)$ and $\cproj_i(C) = (q_i,\rho_i)$, $i=1,2$.

\paragraph{Part \eqref{sync-to-each}.}
 
Let $C' = ((q_1',q_2',R'),\rho')$ and let 
\[
	(q_1,q_2,R) \syncHtr{l}{\sigma} (q_1',q_2',S)
\] 
be the transition inducing $C \tr{a} C'$. We proceed by cases on the rule used to infer this transition:
\begin{itemize}
	\item (\textsc{Reg}): then the transition is inferred from $q_i \htrind{l_i}{\sigma_i}{i} q_i'$, $i=1,2$, such that either $l_1$ or $l_2$ is in $\names$. Suppose, w.l.o.g., $l_1 \in \names$. Then $l = [l_1]_{R^*}$ and $\rho_i(l_1) = \rho([l_1]_{R^*}) = a$, so there is an edge $(q_1,\rho_1) \trind{a}{1} (q_1',\rho_1')$ in the configuration graph of $\tstr_1$. The following chain of equations shows that $\pi_1(C') = (q_1',\rho'_1)$:%we have
	\begin{equation}
		\label{eq:rho}
		\begin{gathered}
			\begin{array}{rl}
				\rho'_1(x) &= \rho_1 (\sigma_1 (x) ) \\
				&= \rho([\sigma_1(x)]_{R^*}) \\
				%&& \text{(by definition of $\rho_i$)}\\
				&= \rho(\sigma_r([x]_{S^*})) \\
				%&& \text{(by definition of $\sigma$)}\\
				&= \rho'([x]_{S^*}) 
				%&& \text{(by definition of $\rho'$)}
			\end{array}
		\end{gathered}
		\tag{$\dagger$}
	\end{equation}
	To prove the existence of an edge $(q_2,\rho_2) \trind{a}{2} (q_2',\rho_2')$ in the configuration graph of $\tstr_2$, we have to consider the following two cases:
	\begin{itemize}
		\item If $l_2 \in \names$, then $\rho_2(l_2) = \rho([l_2]_{R^*}) = \rho([l_1]_{R^*}) = a$, by the rule premise $[l_2]_{R^*} = \{l_1,l_2\}$;
		%
		\item If $l_2 = \star$, then $a$ should be fresh, so we have to check $a \notin \Im(\rho_2)$. Suppose, by contradiction, that there is $x \in \weight{q_2}_2$ such that $\rho_2(x) = a$, then $\rho([x]_{R^*}) = a = \rho([l_1]_{R^*})$, by definition of $\rho$, which implies $[x]_{R^*} = [l_1]_{R^*}$, by injectivity of $\rho$, i.e. $\{l_1,x\} \in [l_1]_{R^*}$, but the premise of the rule states $[l_1]_{R^*} = \{l_1,\star\} \cap \names = \{l_1\}$, so we have a contradiction. 
	\end{itemize}
	Now we have to check $\cproj_2(C') = (q_2',\rho_2')$. Since we have $\rho'_2(x) = (\rho_2 \circ \sigma_2)\sub{a}{\sigma_2^{-1}(\star)}(x)$, for $x \neq \sigma_2^{-1}(\star)$ the equations \eqref{eq:rho} hold. For $x =  \sigma_2^{-1}(\star)$ we have:
	\begin{align*}
		\rho'_2(x) &= (\rho_2 \circ \sigma_2)\sub{a}{x}(x) \\
		&= a \\
		&= \rho([l_1]_{R^*}) \\
		&= (\rho \circ \sigma_r)([x]_{S^*}) \\
		& = \rho'([x]_{S^*})
	\end{align*}	
	%All the equations just apply the definitions of the involved functions.
	%so $\rho([l_1]_R) = \rho([l_2]_R)$, by injectivity of $\rho$.
	%, but this means $, because $[l_1]_R = \{l_1\}$


	\item \allrule: then we have $l=\star$ and the transition is inferred from $q_i \htrind{\star}{\sigma_i}{i} q_i'$, $i=1,2$. Since $a \notin \Im(\rho)$, we also have $a \notin \Im(\rho_i)$, so there are $(q_i,\rho_i) \trind{a}{i} (q_i',\rho_i')$ with $\rho_i' = (\rho_i \circ \sigma_i)\sub{a}{\sigma^{-1}_i(\star)}$, for $i=1,2$. Finally, we have to check that each $\rho_i'(x)$ is as required: if $x \neq\sigma_i^{-1}(\star)$ equations \eqref{eq:rho} hold; for $x=\sigma_i^{-1}(\star)$ we have
	\begin{align*}
		\rho'_i(x) &= (\rho_i \circ \sigma_i)\sub{a}{x}(x) \\
		&= a \\
		&= (\rho \circ \sigma_a) \sub{a}{\sigma_a^{-1}(\star)}(\sigma_a^{-1}(\star)) \\
		&= (\rho \circ \sigma_a) \sub{a}{[x]_{S^*}}([x]_{S^*}) \\
		%&& \text{(by $[\sigma_1^{-1}(\star)]_{S^*} = [\sigma_i(\star)^{-1}]_{S^*}$)} \\
		%& = (\rho \circ \sigma) \sub{a}{[\sigma_i^{-1}(\star)]_{S^*}}([\sigma_i^{-1}(\star)]_{S^*}) && \text{(definition of $R'$)} \\
		&= \rho'([x]_{S^*})
	\end{align*}
	
	%Then there are $q_i \htrind{l_i}{\sigma_i}{i} q_i'$, $i=1,2$. We have two cases:
%	\begin{itemize}
%		\item 
%	\end{itemize}
\end{itemize} 


\paragraph{Part \eqref{each-to-sync}.} 
Since $\tstr_1 \syncp \tstr_2$ is deterministic, there certainly is $C \tr{a} C'$, for any $a \in \names$. This edge, by the previous part of the proof, has a corresponding edge $\cproj_i(C) \trind{a}{i} \cproj_i(C')$, for each $i=1,2$. But then $\cproj_i(C') = C_i$, by determinism of $\tstr_i$.
%Then we must have $\pi(C') = C_i$

% edge in the configuration graph of $\tstr_i$
\qed
\end{proof}

\begin{proof}[of \cref{thm:bool-closure}]
We just consider $\Lang_1 \cap \Lang_2$, the other cases are analogous. Let $\autom_\cap$ be $(\tstr_1 \syncp \tstr_2,\acc_\cap)$; this is a proper \hdma{}, thanks to \cref{rem:syncp-fin-det}. Given $\alpha \in \names^\omega$, let $r_\cap$,$r_1$ and $r_2$ be the runs for $\alpha$ in the configuration graphs of $\autom_\cap,A_1$ and $A_2$, respectively. Then, by \cref{thm:inf-correspondence}, we have $\cproj_i(\Inf(r_\cap)) = \Inf(r_i)$, for each $i=1,2$. From this, and the definition of $\acc_\cap$, we have that $\Inf(r_\cap) \in \acc_\cap$ if and only if $\Inf(r_1) \in \acc_1$ and $\Inf(r_2) \in \acc_2$, i.e.\ $\alpha \in \Lang_{\autom_\cap}$ if and only if $\alpha \in \Lang_{\autom_1}$ and $\alpha \in\Lang_{\autom_2}$.
\qed
\end{proof}
%
\begin{proof}[of \cref{thm:decidable}]
Let $A = (Q,\weight{-},q_0,\rho_0,\htr{}{},\acc)$ be a \hdma{} for $\Lang$. Consider the set $\Sigma_A = \{ (l,\sigma) \mid \exists q,q' \in Q : q \htr{l}{\sigma} q' \}$. This is finite, so we can use it as the alphabet of an ordinary deterministic Muller automaton $M_A = (Q \cup \{\delta\}, q_0,\tr{}_s,\acc)$, where $\delta \notin Q$ is a dummy state, and the transition function is defined as follows: $q \tr{(l,\sigma)}_s q'$ if and only if $q \htr{l}{\sigma} q'$, and $q \tr{(l,\sigma)}_s \delta$ for all other pairs $(l,\sigma) \in \Sigma_A$. Clearly $\Lang_{M_A} = \varnothing$ if and only if $\Lang = \varnothing$, as words in $\Lang_{M_A}$ are sequence of transitions of $A$ that go through accepting states infinitely often, and thus produce a word in $\Lang$, and viceversa. The claim follows by decidability of emptiness for ordinary deterministic Muller automata. Finally, to check equality of languages, observe that the language $(\Lang_1 \cup \Lang_2) \setminus (\Lang_1 \cap \Lang_2 )$ is $\omega$-regular nominal, thanks to \cref{thm:bool-closure}. Then we just have to check its emptiness, which is decidable.
\qed
\end{proof}
%
We give one straightforward lemma about configuration graphs.
%
\begin{lemma}
\label{lem:tr-names}
For all edges $(p_1,\rho_1) \tr{a} (p_2,\rho_2)$ we have $\Im(\rho_2) \subseteq \Im(\rho_1) \cup \{ a \}$.
\end{lemma}
%
We give one additional lemma about $I$ defined in \cref{sec:up-words}.
%
\begin{lemma}
\label{lem:xI}
Given $x \in \dom(\widehat{\sigma})$, suppose there is a positive integer $k$ such that $x = \widehat{\sigma}^k (x)$. Then $x \in I$.
\end{lemma}
\begin{proof}
Suppose $x \notin I$. $I = \widehat{\sigma}(I)$ implies $I = \widehat{\sigma}^k(I)$, so $I \cup \{x\} = \widehat{\sigma}^k(I \cup \{x\})$, but this is against the assumption that $I$ is the largest set satisfying $I = \widehat{\sigma}(I)$.
\qed
\end{proof}




\begin{proof}[of \cref{lem:rho-forget}]
%Suppose $|J_x| \geq n$, otherwise the statement is trivially true. 
Observe that this sequence is such that $x_{kn} \neq x_{k'n}$, for all $k,k' \geq 0$ such that $k \neq k'$. In fact, suppose there are $x_{kn} = x_{k'n}$, with $k < k'$. Then we would have $x_{kn-1} = x_{k'n-1}$, because $\sigma_{n}$ is injective. In general, $x_{kn-l} = x_{k'n-l}$, for $0 \leq l \leq kn$, therefore $x = x_0 = x_{(k'-k)n}$. This means that $\widehat{\sigma}^{(k'-k)}(x) = x$ which, by \cref{lem:xI}, implies $x \in I$, against the hypothesis $x \in T$.

Now, suppose that $J_x = \mathbb{N}$. Then we would have an infinite subsequence $\{x_{jn}\}_{j \in J_x}$ of pairwise distinct names that belong to $\weight{p_0}$, but $\weight{p_0}$ is finite, a contradiction.
\qed
\end{proof}


\begin{proof}[of \cref{lem:idI}]\hfill

\item Let $\pi \colon I \to I$ be the function $\restr{\widehat{\sigma}}{I}$ with its codomain restricted to $I$. Then $\pi$ is an element of the symmetric group on $I$, so it has an order $\id$, that is a positive integer such that $\pi^\id = id_I$. Hence $\restr{ \rrho_\id }{ I } =\restr{ \rrho_0 }{I} \circ \pi^\id = \restr{ \rrho_0 }{I}$.
\qed
\end{proof}

\begin{proof}[of \cref{lem:forgetT}]
Let $\mathcal{J}$ be
\[
	\mathcal{J} := \max \{ |J_x|\mid x \in T \} + 1 .
\]
This gives the number of transitions it takes to forget all the names assigned to $T$. Let $\forg$ be $\lceil \frac{\mathcal{J}}{n} \rceil$. For any $\gamma \geq \forg$, we can choose $v_0,\dots,v_{\gamma-1}$ as any $\gamma$-tuple of words that are recognized by the loop and such that, whenever $l_j = \star$, then $(v_i)_j$ is different from $\Im(\rrho_0)$ and all the previous symbols in $v_0,\dots,v_i$, for all $i=0,\dots,\gamma-1$ and $j=0,\dots,n-1$. Let us verify $\Im(\rrho_\gamma) \cap \rrho_0(T) = \varnothing$ separately on $I$ and $T$ (recall $I \cup T = \weight{p_0})$: we have $\rrho_\gamma(T) \cap \rrho_0(T) = \varnothing$, because all the names assigned to $T$ have been replaced by fresh ones; and we have $\rrho_\gamma(I) = \rrho_0(I)$, so $\rrho_\gamma(I) \cap \rrho_0(T) = \varnothing$.
\qed
\end{proof}


\begin{proof}[of \cref{lem:initT}]
For each name $x \in T$, define a tuple $(x,i,j)$ where $i$ is the index of the transition that consumes the fresh name that will be assigned to $x$, and $j$ is how many traversals of $L$ it takes for this assignment to happen (including the one where the transition $i$ is performed). Formally, $j$ is the smallest integer such that there are $x_{jn},\dots,x_i$ defined as follows
\[
	x_{jn} = x \qquad \sigma_\ul{k+1}(x_{k+1}) = x_k \qquad \sigma_i(x_i) = \star \enspace .
\]
Let $X$ be the set of such tuples and let $\ass := \max \{ j \mid (x,i,j) \in X \}$. Then we can construct $v_0,\dots,v_{\ass-1}$ as follows
\[
	(v_k)_i :=
	\begin{cases}
		\text{$y$ fresh} & l_i = \star \land i \notin \pi_2(X) \\
		\rrho_0(x) & (x,i,\ass - k) \in X
		 \\
		%\trho_0(l_0) & l_0 \neq \star \\
		\trho_{k}(l_i) & l_i \neq \star
	\end{cases}
\]
%
where by $y$ fresh we mean different from elements of $\Im(\trho_0) \cup \Im(\rrho_0)$ and previous symbols in $v_0,\dots,v_{k}$.

The second case in the definition of $(v_k)_i$ is justified as follows. Suppose $\trho_{k,i}$ is the register assignment for $(p_i,\trho_{k,i}) \tr{(v_k)_i} \dots$, then we have to show $(v_k)_i = \rrho_0(x) \notin \Im(\trho_{k,i})$. Suppose, by contradiction, that $\rrho_0(x) \in \Im(\trho_{k,i})$, then by \cref{lem:tr-names} and by how we defined the symbols consumed we have $\rrho_0(x) \in \Im(\trho_0) \cup Y \cup \rrho_0(T')$, for some $T' \subseteq T$, and some set of fresh (in the mentioned sense) names $Y$.
%, where $F$ are all the fresh names consumed so far. In particular, $F$ is made of globally fresh names
%But $F \cap \Im(\rrho_0) = \varnothing$, 
But $\rrho_0(x) \notin Y$, by construction, and $x$ cannot already be in $T'$, because there cannot be two distinct tuples in $X$ that coincide on the first component. Therefore we must have
$\rrho_0(x) \in \Im(\trho_0)$, which implies $\rrho_0(T) \cap \Im(\trho_0) \neq \varnothing$, because $x \in T$, but this contradicts our hypothesis.
% \cap \Im(\rrho_0) = \varnothing$

It is easy to check that this constructions reaches a configuration where all $x \in T$ have been assigned the desired value. 
% $\trho_\ass$ satisfies the statement by construction.\todo{Dire meglio?}
\qed
\end{proof}