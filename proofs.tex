%!TEX root=ndma.tex

\begin{lemma}
\label{lem:xI}
Given $x \in \dom(\widehat{\sigma})$, suppose there is a positive integer $k$ such that $x = \widehat{\sigma}^k (x)$. Then $x \in I$.
\end{lemma}
\begin{proof}
Suppose $x \notin I$. $I = \widehat{\sigma}(I)$ implies $I = \widehat{\sigma}^k(I)$, so $I \cup \{x\} = \widehat{\sigma}^k(I \cup \{x\})$, but this is against the assumption that $I$ is the largest set satisfying $I = \widehat{\sigma}(I)$.
\qed
\end{proof}


\begin{proof}[of \cref{lem:rho-forget}]
Suppose $|J_x| \geq n$, otherwise the statement is trivially true. Observe that this sequence is such that $x_{kn} \neq x_{k'n}$, for all $k,k' \geq 0$ such that $k \neq k'$. In fact, suppose there are $x_{kn} = x_{k'n}$, with $k < k'$. Then we would have $x_{kn-1} = x_{k'n-1}$, because $\sigma_{n}$ is injective. In general, $x_{kn-l} = x_{k'n-l}$, for $0 \leq l \leq kn$, therefore $x = x_0 = x_{(k'-k)n}$. This means that $\widehat{\sigma}^{(k'-k)}(x) = x$ which, by \cref{lem:xI}, implies $x \in I$, against the hypothesis $x \in T$.

Now, suppose that $J_x = \mathbb{N}$. Then we would have an infinite subsequence $\{x_{jn}\}_{j \in J_x}$ of pairwise distinct names that belong to $\weight{p_0}$, but $\weight{p_0}$ is finite, a contradiction.
\qed
\end{proof}


\begin{proof}[of \cref{lem:IT}]\hfill
\begin{enumerate}[(i)]


\item Let $\pi \colon I \to I$ be the function $\restr{\widehat{\sigma}}{I}$ with its codomain restricted to $I$. Then $\pi$ is an element of the symmetric group on $I$, so it has an order $\id$, that is a positive integer such that $\pi^\id = id_I$\todosm{Dovrebbe funzionare sempre con $\id = |I|!$, cioè con l'ordine del gruppo}. Hence $\restr{ \rrho_\id }{ I } =\restr{ \rrho_0 }{I} \circ \pi^\id = \restr{ \rrho_0 }{I}$.


\item
Let $\mathcal{J}$ be
\[
	\mathcal{J} := \max \{ |J_x|\mid x \in T \} + 1 .
\]
This gives the number of transitions it takes to forget all the names stored in $T$. Let $\forg$ be $\lceil \frac{\mathcal{J}}{n} \rceil$. For any $\gamma \geq \forg$, we can choose $v_1,\dots,v_\gamma$ as any $\gamma$-tuple of words that are recognized by the loop and such that, whenever $l_j = \star$, then $(v_i)_j$ is different from $\Im(\rrho_0)$ and all the previous symbols in $v_1,\dots,v_i$, for all $i=1,\dots,\gamma$ and $j=1,\dots,n$. Let us verify $\Im(\rrho_\gamma) \cap \rrho_0(T) = \varnothing$ separately on $I$ and $T$ (recall $I \cup T = \weight{p_0})$: we have $\rrho_\gamma(T) \cap \rrho_0(T) = \varnothing$, because all the names assigned to $T$ have been replaced by fresh ones; and we have $\rrho_\gamma(I) = \rrho_0(I)$, so $\rrho_\gamma(I) \cap \rrho_0(T) = \varnothing$.
\end{enumerate}
\qed
\end{proof}


\begin{proof}[of \cref{lem:initT}]
For each name $x \in T$ define a tuple $(x,i,j)$ where $i$ is the index of the transition where $x$ is allocated and $j$ is the number of loop traversals needed to allocate it, i.e.\ $j$ is the smallest integer such that there are $x_{jn},\dots,x_1$ defined as follows
\[
	x_{jn} = x \qquad \sigma_\ul{k+1}(x_{k+1}) = x_k \qquad \sigma_i(x_1) = \star
\]
Let $X$ be the set of such tuples and let $\ass := \max \{ j \mid (x,i,j) \in X \}$. Then we can construct $v_1,\dots,v_\ass$ as follows
\[
	(v_k)_i :=
	\begin{cases}
		\text{$y$ fresh} & l_i = \star \land i \notin \pi_2(X) \\
		\rrho_0(x) & (x,i,\ass - k + 1)\footnotemark \in X
		 \\
		%\trho_0(l_0) & l_0 \neq \star \\
		\trho_{k-1}(l_i) & l_i \neq \star
	\end{cases}
\]

where by $y$ fresh we mean different from elements of $\Im(\trho_0) \cup \Im(\rrho_0)$ and previous symbols in $v_1,\dots,v_{k}$.

The second case in the definition of $(v_k)_i$ is justified as follows. Suppose $\trho_{k,i}$ is the register assignment for the transition $(p_i,\trho_{k,i}) \tr{(v_k)_i} \dots$, then we have to show $(v_k)_i = \rrho_0(x) \notin \Im(\trho_{k,i})$. Suppose, by contradiction, that $\rrho_0(x) \in \Im(\trho_{k,i})$, then by \cref{lem:tr-names} we have $\rrho_0(x) \in \Im(\trho_0) \cup F$, where $F$ are all the fresh names allocated so far. But $F \cap \Im(\rrho_0) = \varnothing$, by construction, so $\rrho_0(x) \in \Im(\trho_0)$, which implies $\rrho_0(T) \cap \Im(\trho_0) \neq \varnothing$, due to $x \in T$, against our hypothesis.

It is easy to see that $\trho_\ass$ satisfies the statement by construction.
\qed
\end{proof}

\begin{proof}[of \cref{prop:edges-correspondence}]
Let $C = ((q_1,q_2,R),\rho)$ and $\cproj_i(C) = (q_i,\rho_i)$, $i=1,2$.

\paragraph{Part (i).}
 
Let $C' = ((q_1',q_2',R'),\rho')$, $\cproj_i(C) = (q_i,\rho_i)$, for $i=1,2$, and let 
\[
	(q_1,q_2,R) \syncHtr{l}{\sigma} (q_1',q_2',S_2)
\] 
be the transition inducing $C \tr{a} C'$. We proceed by cases on the rule used to infer this transition:
% in $\tstr_1 \syncp \tstr_2$ that induces $((q_1,q_2,R),\rho) \tr{a} ((q_1',q_2',S),\rho')$:
\begin{itemize}
	\item (\textsc{Reg}): then the transition is inferred from $q_i \htrind{l_i}{\sigma_i}{i} q_i'$, $i=1,2$, such that either $l_1$ or $l_2$ is in $\names$. Suppose, w.l.o.g., $l_1 \in \names$. Then $l = [l_1]_{R^*}$ and $\rho_i(l_1) = \rho([l_1]_{R^*}) = a$, so there is an edge $(q_1,\rho_1) \trind{a}{1} (q_1',\rho_1')$ in the configuration graph of $\tstr_1$. The following chain of equations shows that $\pi_1(C') = (q_1',\rho'_1)$:%we have
	\begin{equation}
		\label{eq:rho}
		\begin{gathered}
			\begin{array}{rl}
				\rho'_1(x) &= \rho_1 (\sigma_1 (x) ) \\
				&= \rho([\sigma_1(x)]_{R^*}) \\
				%&& \text{(by definition of $\rho_i$)}\\
				&= \rho(\sigma([x]_{S^*})) \\
				%&& \text{(by definition of $\sigma$)}\\
				&= \rho'([x]_{S^*}) 
				%&& \text{(by definition of $\rho'$)}
			\end{array}
		\end{gathered}
		\tag{$\dagger$}
	\end{equation}
	To prove the existence of an edge $(q_2,\rho_2) \trind{a}{2} (q_2',\rho_2')$ in the configuration graph of $\tstr_2$, we have to consider the following two cases:
	\begin{itemize}
		\item If $l_2 \in \names$, then $\rho_2(l_2) = \rho([l_2]_{R^*}) = \rho([l_1]_{R^*}) = a$, by the rule premise $[l_2]_{R^*} = \{l_1,l_1\}$;
		%
		\item If $l_2 = \star$, then $a$ should be fresh, so we have to check $a \notin \Im(\rho_2)$. Suppose, by contradiction, that there is $x \in \weight{q_2}_2$ such that $\rho_2(x) = a$, then $\rho([x]_{R^*}) = a = \rho([l_1]_{R^*})$, by definition of $\rho$, which implies $[x]_{R^*} = [l_1]_{R^*}$, by injectivity of $\rho$, i.e. $\{l_1,l_2\} \in [l_1]_{R^*}$, but the premise of the rule states $[l_1]_R = \{l_1,\star\} \cap \names = \{l_1\}$, hence the contradiction. 
	\end{itemize}
	Now we have to check that $\rho_2'$ satisfies the claim. Since we have $\rho'_2(x) = (\rho_2 \circ \sigma_2)\sub{a}{\sigma_2^{-1}(\star)}(x)$, for $x \neq \sigma_2^{-1}(\star)$ the equations \eqref{eq:rho} hold. For $x =  \sigma_2^{-1}(\star)$ we have:
	\begin{align*}
		\rho'_2(x) &= (\rho_2 \circ \sigma_2)\sub{a}{x}(x) \\
		&= a \\
		&= \rho([l_1]_{R^*}) \\
		&= (\rho \circ \sigma)([x]_{S^*}) \\
		& = \rho'([x]_{S^*})
	\end{align*}	
	%All the equations just apply the definitions of the involved functions.
	%so $\rho([l_1]_R) = \rho([l_2]_R)$, by injectivity of $\rho$.
	%, but this means $, because $[l_1]_R = \{l_1\}$


	\item \allrule: then we have $l=\star$ and the transition is inferred from $q_i \htrind{\star}{\sigma_i}{i} q_i'$, $i=1,2$. Since $a \notin \Im(\rho)$, we also have $a \notin \Im(\rho_i)$, so there are $(q_i,\rho_i) \trind{a}{i} (q_i',\rho_i')$ with $\rho_i' = (\rho_i \circ \sigma_i)\sub{a}{\sigma^{-1}_i(\star)}$, for $i=1,2$. Finally, we have to check that each $\rho_i'(x)$ is as required: if $x \neq\sigma_i^{-1}(\star)$ equations \eqref{eq:rho} hold; for $x=\sigma_i^{-1}(\star)$ we have
	\begin{align*}
		\rho'_i(x) &= (\rho_i \circ \sigma_i)\sub{a}{x}(x) \\
		&= a \\
		&= (\rho \circ \sigma_\star) \sub{a}{\sigma_\star^{-1}(\star)}(\sigma_\star^{-1}(\star)) \\
		&= (\rho \circ \sigma_\star) \sub{a}{[x]_{S^*}}([x]_{S^*}) \\
		%&& \text{(by $[\sigma_1^{-1}(\star)]_{S^*} = [\sigma_i(\star)^{-1}]_{S^*}$)} \\
		%& = (\rho \circ \sigma) \sub{a}{[\sigma_i^{-1}(\star)]_{S^*}}([\sigma_i^{-1}(\star)]_{S^*}) && \text{(definition of $R'$)} \\
		&= \rho'([x]_{S^*})
	\end{align*}
	
	%Then there are $q_i \htrind{l_i}{\sigma_i}{i} q_i'$, $i=1,2$. We have two cases:
%	\begin{itemize}
%		\item 
%	\end{itemize}
\end{itemize} 


\paragraph{Part (ii).}

Let $(q_i,\rho_i) \trind{a}{i} C_i = (q_i',\rho_i')$ be the edges in the statement, for $i=1,2$. We have to find the transitions in $\tstr_1$ and $\tstr_2$ yielding these edges, and use them to compute a transition of $\tstr_1 \syncp \tstr_2$ and the corresponding edge in the configuration graph. We have three cases: 
\begin{itemize} 
	\item Suppose there are $l_1 \in \weight{q_1}_1$ and $l_2 \in \weight{q_2}_2$ such that $\rho_1(l_1) = \rho_j(l_2) = a$, then there are two transitions $q_1 \htrind{l_1}{\sigma_1}{1} q_1'$ and $q_2 \htrind{l_2}{\sigma_2}{2} q_2'$. Since $[l_1]_{R^*} = \{l_1,l_2\}$, we can apply \regrule{} and get $(q_1,q_2,R) \syncHtr{[l_1]_{R*}}{\sigma} (q_1',q_2',S)$. Since $\rho([l_1])_{R^*} = a$, this transition induces the edge $((q_1,q_2,R),\rho) \tr{a} ((q_1',q_2',S),\rho')$, where, for $x' \in \weight{q_i'}_i$:
	\begin{align*}
		\rho'([x]_{S^*}) &= \rho(\sigma([x]_{S^*})) \\
		&= \rho([\sigma_i(x)]_{R^*}) \\
		&= \rho_i(\sigma_i(x)) \\
		&= \rho_i'(x)
	\end{align*}
	%
	\item Suppose $l_i = \star$ and $l_j \in \weight{q_j}_j$.
	%
	\item Suppose $l_i = l_j = \star$.
	%
\end{itemize}


\qed
\end{proof}
