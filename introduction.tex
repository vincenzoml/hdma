%!TEX root=ndma.tex

% Explain what kind of languages do we get (refer to the example in the paper)

Languages of infinite words are of paramount importance in logics and computer science. Their usage scenarios range from theoretical proofs of decidability for fragments of first-order to applications of relevant practical impact, such as model checking and learning of logical properties. Just as in the classical case, these languages are typically defined on finite alphabets. However, there are cases in which the alphabet is infinite; think e.g., about \emph{data words} \cite{TODO}, or \emph{nominal calculi} \cite{Pi-calculus}. Languages of \emph{finite} words over infinite alphabets have already been studied in the literature, in the form of \emph{register automata} \cite{TODO}. It is nowadays clear that register automata, and languages of infinite alphabets, are also expressible as automata over \emph{nominal sets}, which are in turn equivalent  to history-dependent automata (see \cite{CianciaTuostoTR} for an introduction). 

More recently, the paper \cite{MikLICS} initiated a thorough investigation on the languages that are expressible using variants of nominal sets, and on \emph{nominal computation} in general. The same point of view led to the developments described in \cite{MikPOPL12}, and \cite{PittsPOPL13}. Nominal sets introduce the key notion of \emph{finite support}, that can be roughly explained as a finite memory property with respect to the symbols that appear on a word. From the automata-theoretic perspective, languages of finite words over infinite (nominal) alphabets are treated in a satisfactory way by just defining an automaton as an \emph{orbit-finite} set of states \cite{CiaMonIC}, equipped with an \emph{equivariant} transition relation, and equivariant acceptance condition. Each finite word is finitely supported, thus the set of all words forms a nominal set. 

The case of infinite words over nominal alphabets is more problematic, as an infinite word over an infinite alphabet is generally not finitely supported. Consider a machine that reads any symbol from an infinite, countable alphabet, and never stores it. Clearly, such a machine has finite (empty) memory. The set of its traces is simply described as the set of all infinite words over the alphabet. However, in the language we have various species of words. Some of them are finitely supported; for example, words that consist of the infinite repetition of a finite word. Some others are not finitely supported, such as the word enumerating all the symbols of the alphabet. Such words lay inherently out of the settting of nominal sets. However, the existence of these words does not give to the language infinite memory. More precisely, if we only consider finite-memory machines, problematic words can not be singled out; rather, their presence in a language can not be separated from the presence of an infinite number of other words. The simple machine we described accepts all possible infinite words. A machine that would accept only the enumeration of all symbols, instead, would require infinite memory. 

The aim of this work is to translate the intuitions in the previous paragraphs into precise mathematical terms, in order to define a class of languages of infinite words over infinite alphabets that possess finite-memory properties. We extend the automata over nominal sets of \cite{MikLICS}, also used in \cite{CianciaTuostoTR}, to handle infinite words, by imposing a Muller-automata-alike condition over the \emph{orbits} (not the states!) of the automata. By doing so, it turns out that our languages not only are finite-memory, but they retain computational properties, such as closure under boolean operations and decidability of emptyness (thus, containment and equivalence). Moreover, a somewhat unexpected result is that the obtained languages are determined by their \emph{ultimately periodic} fragments, just as in the classical case. This clarifies why it is not possible to single-out problematic streams, as we explained above. Being determined by ultimately periodic fragments is fundamental for classical automata over infinite words, whose consequences have probably not yet been explored in full. For example, this property makes it possible to learn languages of infinite words \cite{Emerson}, or to find canonical representatives up-to language equivalence, in a coalgebraic flavour \cite{CV12}. 



