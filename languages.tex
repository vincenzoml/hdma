%!TEX root=ndma.tex

Traditionally, automata can be deterministic or non-deterministic. In order to extend previous results on closure under complementation and decidability, one needs to work with deterministic structures; therefore, in this paper, we introduce directly the deterministic structures.

\begin{definition}
 A \emph{nominal deterministic Muller automaton} (nDMA) is a tuple $(Q,\tr{},q_0,\acc)$ where:
 
  \begin{itemize}
  \item $Q$ is an orbit-finite nominal set of \emph{states};
  
  \item $q_0 \in Q$ is the \emph{initial state};
  
  \item $\acc \subseteq \Pow(\Orb(Q))$ is a set of sets of orbits, intended to be used as an accepting condition in the style of Muller automata.
  
  \item $\htr{}{}$ is the \emph{transition relation}, made up of triples $q_1 \tr{l} q_2$, having \emph{source} $q_1$, \emph{target} $q_2$, \emph{label} $l \in \names$;
  
  \item the transition relation is \emph{deterministic}, that is, for each $q \in Q$ and label $l \in \names$ there is exactly one transition with source $q$ and label $l$.
 \end{itemize}
\end{definition}

