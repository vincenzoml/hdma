%!TEX root=ndma.tex
\newcommand{\compl}[1]{\overline{#1}}
%
%In this section we discuss closure of $\omega$-regular nominal languages under boolean operations, and decidability of emptiness and language equality.
%
Let $\Lang_1$ and $\Lang_2$ be $\omega$-regular nominal languages, and let $\autom_1 = (\tstr_1,\acc_1)$  and $\autom_2 = (\tstr_2,\acc_2)$ be automata for these languages, where $\tstr_1$ and $\tstr_2$ are the underlying transition structures. By \cref{thm:inf-correspondence} above, 
we are now able to show that constructing the automaton for a boolean combination of $\Lang_1$ and $\Lang_2$ amounts to defining an appropriate accepting set for $\tstr_1 \syncp \tstr_2$.


\begin{theorem}
%\begin{definition} 
Using the transition structure $\tstr_1 \syncp \tstr_2$, define the accepting conditions $\acc_\cap = \{ S \subseteq \syncQ \mid \pi_1(S) \in \acc_1 \land \pi_2(S) \in \acc_2 \}$, $\acc_\cup = \{ S \subseteq \syncQ \mid \pi_1(S) \in \acc_1 \lor \pi_2(S) \in \acc_2 \} $ and $\acc_{\compl{\Lang_1}} = \Pow(Q_1) \setminus \acc_1$, where $Q_1$ are the states of $\autom_1$.
%Let $\Lang_1$ and $\Lang_2$ languages of $\hdmas$ having transition structures $\tstr_1$ and $\tstr_2$ and acceptance condition $\acc_1$ and $\acc_2$. The automata for intersection and union are obtained by equipping the structure $\tstr_1 \syncp \tstr_2$ with accepting conditions $\acc_\cap$ and $\acc_\cup$, respectively, defined as:
%
% \begin{align*}
% 	\acc_\cap &:= \{ S \subseteq \syncQ \mid \pi_1(S) \in \acc_1 \land \pi_2(S) \in \acc_2 \} 
% 	\\
% 	\acc_\cup &:= \{ S \subseteq \syncQ \mid \pi_1(S) \in \acc_1 \lor \pi_2(S) \in \acc_2 \} 
% 	\\
% 	%\acc_\cap &:= \{ S \subseteq \syncQ \mid \pi_1(S) \notin \acc_1 \} 
% 	\acc_{\compl{\Lang_1}} &:= \Pow(Q_1) \setminus \acc_1 \qquad \text{where $Q_1$ are the states of $\autom_1$.}
% %	\\
% %	\acc_{\setminus} &:= \bigcup_{S_1 \in \acc_1} \{\{ (q_1,q_2,R) \in \syncQ \mid q_1 \in S_1 \land \forall S_2 \in \acc_2 : q_2 \notin S_2 \}\}
% \end{align*}
%
%
%The automaton for $\compl{\Lang_1}$ can be obtained by complementing $\acc_1$ in $\Pow(Q_1)$, where $Q_1$ are the states of the corresponding \hdma.
%but also as the automaton for $\names^\omega \setminus \Lang_1$ (see \cref{fig:hd-names-omega} for the automaton accepting $\names^\omega$). 
%\end{definition}
The obtained \hdma{}s accept, respectively, $\Lang_1 \cap \Lang_2$, $\Lang_1 \cup \Lang_2$, and $\compl{\Lang_1}$.
\label{thm:bool-closure}
\end{theorem}
%%
%\noindent Finally, we can state our decidability results.
%
\begin{theorem}
\label{thm:decidable}
Emptiness and, as a corollary, equality of $\omega$-regular nominal languages are decidable.
\end{theorem}
