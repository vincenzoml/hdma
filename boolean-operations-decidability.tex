%!TEX root=ndma.tex
\newcommand{\compl}[1]{\overline{#1}}
 
In this section we discuss closure of $\omega$-regular nominal languages under boolean operations, and decidability of emptiness and language equality.

Let $\Lang_1$ and $\Lang_2$ be $\omega$-regular nominal languages, and let $\autom_1 = (\tstr_1,\acc_1)$  and $\autom_2 = (\tstr_2,\acc_2)$ be automata for these languages, where $\tstr_1$ and $\tstr_2$ are the underlying transition structures.
The crucial tool is \cref{thm:inf-correspondence}: it says that constructing an automaton for a boolean combination of $\Lang_1$ and $\Lang_2$ amounts to defining an appropriate accepting set for $\tstr_1 \syncp \tstr_2$.

The accepting sets for intersection, union and difference are:
%
\begin{align*}
	\acc_\cap &:= \bigcup_{S_1 \in \acc_1,S_2 \in \acc_2 } \{\{ (q_1,q_2,R) \in \syncQ \mid q_1 \in S_1 \land q_2 \in S_2 \}\} 
	\\
	\acc_\cup &:= \bigcup_{S_1 \in \acc_1,S_2 \in \acc_2 } \{\{(q_1,q_2,R) \in \syncQ \mid q_1 \in S_1 \lor q_2 \in S_2 \}\} 
	\\
	\acc_{\setminus} &:= \bigcup_{S_1 \in \acc_1} \{\{ (q_1,q_2,R) \in \syncQ \mid q_1 \in S_1 \land \forall S_2 \in \acc_2 : q_2 \notin S_2 \}\}
\end{align*}
%
The automaton for $\compl{\Lang_1}$ can be obtained by complementing $\acc_1$, but also as the automaton for $\names^\omega \setminus \Lang_1$ (see \cref{ex:nomega-hdma} for the automaton accepting $\names^\omega$).


\begin{theorem}
$\Lang_1 \cap \Lang_2$, $\Lang_1 \cup \Lang_2$, $\Lang_1 \setminus \Lang_2$, $\compl{\Lang_1}$ are $\omega$-regular nominal languages.
\label{thm:bool-closure}
\end{theorem}
%%
Now we give decidability results.
%
\begin{theorem}
Emptiness of $\Lang$ is decidable
\label{thm:emptiness}
\end{theorem}

\begin{theorem}
It is decidable whether $\Lang_1 = \Lang_2$.
\end{theorem}

\begin{proof}
The language $(\Lang_1 \cup \Lang_2) \setminus (\Lang_1 \cap \Lang_2 )$ is $\omega$-regular nominal, thanks to \cref{thm:bool-closure}. Then we just have to check its emptiness, which is decidable by \cref{thm:emptiness}.
\qed
\end{proof}
