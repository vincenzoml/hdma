%!TEX root=ndma.tex
\newcommand{\compl}[1]{\overline{#1}}
 
In this section we discuss closure of $\omega$-regular nominal languages under boolean operations, and decidability of emptiness and language equality.

Let $\Lang_1$ and $\Lang_2$ be $\omega$-regular nominal languages, and let $\autom_1 = (\tstr_1,\acc_1)$  and $\autom_2 = (\tstr_2,\acc_2)$ be automata for these languages, where $\tstr_1$ and $\tstr_2$ are the underlying transition structures.
The crucial tool is \cref{thm:inf-correspondence}: it says that constructing an automaton for a boolean combination of $\Lang_1$ and $\Lang_2$ amounts to defining an appropriate accepting set for $\tstr_1 \syncp \tstr_2$.

The accepting sets for intersection, union and difference are:
%
\begin{align*}
	\acc_\cap &:= \bigcup_{S_1 \in \acc_1,S_2 \in \acc_2 } \{\{ (q_1,q_2,R) \in \syncQ \mid q_1 \in S_1 \land q_2 \in S_2 \}\} 
	\\
	\acc_\cup &:= \bigcup_{S_1 \in \acc_1,S_2 \in \acc_2 } \{\{(q_1,q_2,R) \in \syncQ \mid q_1 \in S_1 \lor q_2 \in S_2 \}\} 
	\\
	\acc_{\setminus} &:= \bigcup_{S_1 \in \acc_1} \{\{ (q_1,q_2,R) \in \syncQ \mid q_1 \in S_1 \land \forall S_2 \in \acc_2 : q_2 \notin S_2 \}\}
\end{align*}
%
The automaton for $\compl{\Lang_1}$ can be obtained by complementing $\acc_1$, but also as the automaton for $\names^\omega \setminus \Lang_1$ (see \cref{fig:nomega-automaton} for the automaton accepting $\names^\omega$).


\begin{theorem}
$\Lang_1 \cap \Lang_2$, $\Lang_1 \cup \Lang_2$, $\Lang_1 \setminus \Lang_2$, $\compl{\Lang_1}$ are $\omega$-regular nominal languages.
\label{thm:bool-closure}
\end{theorem}
%
\begin{proof}
We just consider $\Lang_1 \cap \Lang_2$, the other cases are analogous. Let $\autom_\cap$ be $(\tstr_1 \syncp \tstr_2,\acc_\cap)$; this is a proper \hdma{}, thanks to \cref{rem:syncp-fin-det}. Given $\alpha \in \names^\omega$, let $r^\alpha_\cap$,$r_1^\alpha$ and $r_2^\alpha$ be the runs for $\alpha$ in the configuration graphs of $\autom_\cap,A_1$ and $A_2$, respectively. Then, by \cref{thm:inf-correspondence}, we have $\cproj_i(\Inf(r^\alpha_\cap)) = \Inf(r^\alpha_i)$, for each $i=1,2$. From this, and the definition of $\acc_\cap$, we have that $\Inf(r^\alpha_\cap) \in \acc_\cap$ if and only if $\Inf(r^\alpha_1) \in \acc_1$ and $\Inf(r^\alpha_2) \in \acc_2$, i.e.\ $\alpha \in \Lang_{\autom_\cap}$ if and only if $\alpha \in \Lang_{\autom_1}$ and $\alpha \in\Lang_{\autom_2}$.
\qed
\end{proof}
%
Now we give decidability results.
%
\begin{theorem}
Emptiness of $\Lang$ is decidable
\label{thm:emptiness}
\end{theorem}
\begin{proof}
Let $A = (Q,\weight{-},q_0,\rho_0,\htr{}{},\acc)$ be a \hdma{} for $\Lang$. Consider the set $\Sigma_A = \{ (l,\sigma) \mid \exists q,q' \in Q : q \htr{l}{\sigma} q' \}$. This is finite, so we can use it as the alphabet of an ordinary deterministic Muller automaton $M_A = (Q \cup \{\delta\}, q_0,\tr{}_s,\acc)$, where $\delta \notin Q$ is a dummy state, and the transition function is defined as follows: $q \tr{(l,\sigma)}_s q'$ if and only if $q \htr{l}{\sigma} q'$, and $q \tr{(l,\sigma)}_s \delta$ for all other pairs $(l,\sigma) \in \Sigma_A$. Clearly $\Lang_{M_A} = \varnothing$ if and only if $\Lang = \varnothing$, as words in $\Lang_{M_A}$ are sequence of transitions of $A$ that go through accepting states infinitely often, and thus produce a word in $\Lang$, and viceversa. The claim follows by decidability of emptiness for ordinary deterministic Muller automata.
\end{proof}

\begin{theorem}
It is decidable whether $\Lang_1 = \Lang_2$.
\end{theorem}

\begin{proof}
The language $\Lang = (\Lang_1 \cup \Lang_2) \setminus (\Lang_1 \cap \Lang_2 )$ is $\omega$-regular nominal, thanks to \cref{thm:bool-closure}. Then we just have to check the emptiness of $\Lang$, which is decidable by \cref{thm:emptiness}.
\end{proof}
