%!TEX root=ndma.tex

In order to introduce the presented topic, in this section we discuss an application of nominal automata to distributed systems. Consider an idealised \emph{peer-to-peer} system where each peer receives queries from an arbitrary, unbounded number of other peers, represented by an infinite set of unique identifiers. Each peer buffers requests in a finite queue. Then one query is selected, among the buffered ones, and is served by establishing a temporary connection with the target. Peers are not normally supposed to terminate, thus their relevant properties ought to predicate on infinite words. On the other hand, actions executed by peers carry information about other peers, which are drawn from an infinite set, therefore the symbols constituting words are infinite. Finally, each peer has finite memory. This is the key to maintain decidability, and is mathematically modelled by the notion of \emph{finite support} in nominal automata. Once established that our languages are of infinite words over an infinite alphabet, but retain a finite memory property, we can attempt to use automata to characterise properties of local peers in a global environment. 
% and serves them in a first-come-first-served order. The number of unserved queries at each time is bounded, say at most $n$. When a query is selected to be served, the peer established a temporary connection with the target peer. We assume that peers have a fair behavior, i.e.\ they do not accept other queries from peers that have already sent one, if this has yet to be served.
%We want to model such a peer as an automaton, as succinct as possible, i.e.\ with a small number of states and transitions. 
We assume peers have four kinds of observable actions: 
\begin{itemize}
	\item arrival of a new query from $p$, denoted $q(p)$; 
	\item selection of a query to serve, denoted $s(p)$; 
	\item connection to $p$ to serve its request, denoted $c(p)$; 
	\item disconnection from $p$, denoted $d(p)$.
\end{itemize}
%


Our main example will be the automata-theoretic specification of a ``first come first serve'' peer selection policy for queries, named \emph{FCFS fair policy}; we shall also discuss a policy that keeps into account a number of locally identified ``friend peers'' taking priority over the others, that we call \emph{friend policy}. 
%Then, using results from \autoref{sec:boolean-operations-decidability}, the negation of these automata can be intersected with a user-defined peer model: if the resulting automaton has empty language, which is a decidable check, then the peer complies with those policies.
%one can give an automaton representing a peer and check that the intersection of 


%The crucial idea is using symbolic actions:

\paragraph{FCFS fair policy.}
We want to model query selection in a FCFS style. Moreover, we want to capture fairness: queries from already buffered peers are discarded. We assume that the buffer has size $n$. 


\begin{figure}[t]
\centering
\begin{tikzpicture}[->,shorten >=1pt,auto,node distance=2.8cm,semithick,initial text={},scale=0.7, every node/.style={scale=0.7}]
	
%  \node[state,initial] (q0) {$q_0$}; 
%  \node[state,right=9ex of q0] (q1) {$q_1$};
%  \node[state,right=9ex of q1] (q2) {$q_2$};
%
%  \path (q0) edge [loop below]  node[inner sep=1pt] (star) {$q(\star),enq$} (q0);
%  \path (q0) edge node[inner sep=1pt] (s1) {$s(1),id$} (q1);
%  \path (q1) edge node[inner sep=1pt] (s1) {$c(1),id$} (q2);
%  \path (q2) edge [above,out=135,in=45] node[inner sep=1pt] (s1) {$d(1),deq$} (q0);

  \node[state,initial] (q0) {$q_0$}; 
  \node[state,right=10ex of q0] (q1) {$q_1$};
  \node[state,right=10ex of q1] (q2) {$q_2$};
  \node[right=10ex of q2] (dots) {$\dots$};
  \node[state,right=10ex of dots] (qn) {$q_n$};

  \node[state,below=6ex of q1] (s1) {$s_1$};
  \node[state,below=6ex of q2] (s2) {$s_2$};
  \node[below=10ex of dots] (dotss) {$\dots$};
  \node[state,below=6ex of qn] (sn) {$s_n$};

  \node[state,below=6ex of s1] (c1) {$c_1$};
  \node[state,below=6ex of s2] (c2) {$c_2$};
  \node[below=10ex of dotss] (dotsss) {$\dots$};
  \node[state,below=6ex of sn] (cn) {$c_n$};

  \path (q0) edge node[inner sep=1pt,above] {\lab{q(\star)}} node[inner sep=1pt,below] {\lab{enq_1}} (q1);
  \path (q1) edge node[inner sep=1pt,above] {\lab{q(\star)}} node[inner sep=1pt,below] {\lab{enq_2}} (q2);
  \path (q2) edge node[inner sep=1pt,above] {\lab{q(\star)}} node[inner sep=1pt,below] {\lab{enq_3}}  (dots);
  \path (dots) edge node[inner sep=1pt,above] {\lab{q(\star)}} node[inner sep=1pt,below] {\lab{enq_n}} (qn);

%  \path (q2) edge [above,out=135,in=45] node[inner sep=1pt] (s1) {$d(1),deq$} (q0);

  \path (q1) edge node[inner sep=1pt,left] {\lab{s(1)}} node[inner sep=1pt,right] {\lab{id}} (s1);
  \path (q2) edge node[inner sep=1pt,left] {\lab{s(1)}} node[inner sep=1pt,right] {\lab{id}} (s2);
  \path (qn) edge node[inner sep=1pt,left] {\lab{s(1)}} node[inner sep=1pt,right] {\lab{id}} (sn);
  \path (qn) edge[loop] node[inner sep=1pt,above] {\lab{q(\star)}} node[inner sep=1pt,below] {\lab{id}} (qn);

  \path (s1) edge[bend left] node[inner sep=1pt,left] {\lab{c(1)}} node[inner sep=1pt,right] {\lab{id}} (c1);
  \path (s2) edge[bend left] node[inner sep=1pt,left] {\lab{c(1)}} node[inner sep=1pt,right] {\lab{id}} (c2);
  \path (sn) edge[bend left] node[inner sep=1pt,left] {\lab{c(1)}} node[inner sep=1pt,right] {\lab{id}} (cn);

  \path (c1) edge node[inner sep=1pt,left] {\lab{d(1)}} node[inner sep=1pt,right] {\lab{deq_1}} (q0);
  \path (c2) edge node[inner sep=1pt,left] {\lab{d(1)}} node[inner sep=1pt,right] {\lab{deq_2}} (q1);


%  \path (q2)
%  \path (qn)
	

\end{tikzpicture}
\caption{Automaton for the FCFS policy.}
\label{fig:fcfs}
\end{figure}
%
The automaton is shown in \autoref{fig:fcfs}. Each state $q_i,s_i,c_i$ is equipped with $i$ registers, for $i=i,\dots,n$, and $q_0$ has no registers. Registers are local to states (we will discuss this aspect throughout the paper), thus each transition is equipped with a function expressing how registers in the target state take values from those of the source state.
\[
	enq_i(x) = 
	\begin{cases}
		\star & x = i \\
		x & x < i
	\end{cases}
	\qquad
%	shift(x) =
%x	\begin{cases}
%		\star & x = n \\
%		x+1 & x< n
%	\end{cases}
	\qquad
	deq_i(x) = x + 1
\]
%
The intuition is that transitions from $q_i$ to $q_{i+1}$ correspond to buffering a ``fresh'' query, from a peer whose identiy is not already known, and thus it is stored in register $i+1$. The loop on $q_n$ discards new peers when the buffer is full. A transition from $q_i$ to $s_i$ picks \marginpar{V: Potremmo accorpare gli stati $s_i$ e $c_i$ per ogni $i$? Mi sembra renda più leggibile il tutto}the query $p$ in register $1$, that is always the oldest one; transition $s_i$ to $c_i$ establishes a connection to $p$; transition $c_i$ to $q_{i-1}$ removes $p$ from the buffer, and shifts the registers' content so that now register $1$ contains the query that arrived right after the one of $p$. Our automaton should (1) be deterministic and (2) have a Muller-style accepting condition. For (1), we assume each state has all possible outgoing transitions: those not shown in \autoref{fig:fcfs} are assumed to go to a sink state. For (2), we take all subsets of the states, excluding the sink one, as Muller sets; that is: behaviors that go through states and transitions depicted in \autoref{fig:fcfs} are all accepted. All these notions will be clearer after we formally introduce nominal Muller automata.
%Our automata have a Muller-style accepting condition
%The acceptance conditions contains all subsets of the states of this automaton, excluding the sink one.

%

\paragraph{Friend queries.} A \emph{friend query} is a query coming from a ``friend'' peer, which should be served as soon as possible, that is: after the current query has been served. To model such scenarios, one can introduce an action $q_f(p)$ to model a friend query from $p$. An automaton that correctly handles such queries can be obtained from the one of \autoref{fig:fcfs} as follows: we add a transition from $q_i$ to $q_{i+1}$, for each $i=0,\dots,n-1$, labelled with $q_f(\star)$ and with the map $top_i$, sending $1$ to $\star$ and $x > 1$ to $x-1$; furthermore, a looping transition on $q_n$ is added with label $q_f(\star)$ and map $id$, which discards friend queries when the buffer is full. Intuitively, $top_i$ always stores the friend query in register $1$, so that transitions $s(1)$ will always pick it, and shifts the priority of all the other peers.\todo[size=\tiny]{Qui assumo che essere ``friend'' è una proprietà del peer, cioè un predicato unario. Se dovessi catturare una relazione di amicizia probabilmente avrei bisogno di registri costanti che contengono gli identificatori degli amici, oppure di modellare relazioni di amicizia con grafi. V: certo, ma dal punto di vista locale, questa cosa non importa, è come se ci fosse un compilatore, diciamo così, che traduce il livello globale in quello locale}
%by modifying the one of \autoref{fig:fcfs} as follows: we add states $f_i$ ($i=1,\dots,n$)
