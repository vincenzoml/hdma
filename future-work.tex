This work is an attempt to provide a simple definition that merges the theories of nominal automata and $\omega$-regular languages, retaining effective closure under boolean operations, and decidability of emptiness, and language equivalence. We sketch some possible future directions. A very relevant application of formal verification in the presence of fresh resources could be model-checking of nominal process calculi. However, the presented theory only accommodates the deterministic case; undecidability issues arise for non-deterministic systems. Future work will be directed to identify (fragments of) nominal calculi that retain decidability. For this, one needs to limit not only non-determinism, but also parallel composition (again, again, decidability may be an issue otherwise). A calculus that can be handled by the current theory is the determinstic, finite-control pi-calculus. It is nowadays well known that nominal sets correspond to presheaves over finite sets and injections that are \emph{sheaves} with respect to the atomic topology (the so-called \emph{Schanuel topos}), and that HD-automata correspond to coalgebras on such sheaves. By changing the index category of sheaves one obtains different kinds of nominal sets \cite{CianciaKM10}, and different classes of HD-automata. Since \hdmas{} are based on HD-automata, this correspondence seems relevant also for our work. For instance, by taking sheaves over graphs \cite{SammartinoPhD}, one could express complex relations among symbols in the alphabet, and require that, infinitely often, one encounters a symbol which is related in a certain way to a number of its predecessors. Furthermore, recall that automata correspond to logic formulae: \hdmas{} could be used to represent logic formulae with binders; it would also be interesting to investigate the relation with first-order logic on nominal sets \cite{Bojanczyk13}. There may be different logical interpretations of \hdmas, where causality or dependence \cite{Vnnen07,Galliani12} between events are made explicit. Finally, extending the two-sorted coalgebraic representation of Muller automata introduced in \cite{CV12} to \hdmas{} would yield canonical representative of automata up to language equivalence.



%%%%%%%%%%% OLD VERSION
%!TEX root=ndma.tex
% This work is an attempt to provide a simple definition that merges the theories of nominal automata and $\omega$-regular languages, retaining effective closure under boolean operations, decidability of emptiness and language equivalence, and determinacy by ultimately periodic words. We sketch some possible future directions. It is well known that nominal sets correspond to presheaves over finite sets and injections that are \emph{sheaves} with respect to the atomic topology (the so-called \emph{Schanuel topos}), and that HD-automata correspond to coalgebras on such sheaves. By changing the index category of sheaves one obtains different kinds of nominal sets \cite{CianciaKM10}, and different classes of HD-automata. Since \hdmas{} are based on HD-automata, this correspondence seems relevant also for our work. For instance, by taking sheaves over graphs \cite{SammartinoPhD}, one could express complex relations among symbols in the alphabet, and require that, infinitely often, one encounters a symbol which is related in a certain way to a number of its predecessors. Furthermore, recall that automata correspond to logic formulae: \hdmas{} could be used to represent logic formulae with binders; it would also be interesting to investigate the relation with first-order logic on nominal sets \cite{Bojanczyk13}. There may be different logical interpretations of \hdmas, where causality or dependence \cite{Vnnen07,Galliani12} between events are made explicit. Finally, extending the two-sorted coalgebraic representation of Muller automata introduced in \cite{CV12} to \hdmas{} would yield canonical representative of automata up to language equivalence.
