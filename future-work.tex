This work is an attempt to provide a simple definition that merges the theories of nominal automata and $\omega$-regular languages, retaining effective closure under boolean operations, and determinacy by ultimately periodic words. We sketch some possible future directions. It is well known that nominal set correspond to sheaves over the \emph{atomic topology}, which form a subcategory of a presheaf category indexed by finite sets \cite{fabio,staton}. By changing the index category, one may express more complex relations among symbols in the alphabet, such as, graph-based network topologies \cite{matteo}. Extending nDMA to work in such a richer setting seems relevant. For example, using graphs, one could predicate on arbitrary relations between symbols, and require that infinitely often, one passes by a symbol which is related in a certain way to a number of its predecessors. Furthermore, recall that automata correspond to logic formulae. Nominal Muller automata could be put in correspondence with logic formulae with binders. However, there may be different logical interpretations of automata, where causality or dependence between events are made explicit. Finally, extending the two-sorted setting of \cite{CV12} to hDMAs would enhance the defined framework with canonical representatives of automata, up-to language equivalence.
