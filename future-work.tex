%!TEX root=ndma.tex
This work is an attempt to provide a simple definition that merges the theories of nominal automata and $\omega$-regular languages, retaining effective closure under boolean operations, decidability of emptiness and language equivalence, and determinacy by ultimately periodic words. We sketch some possible future directions. It is well known that nominal sets correspond to presheaves over finite sets and injections that are \emph{sheaves} with respect to the atomic topology (the so-called \emph{Schanuel topos}), and that HD-automata correspond to coalgebras on such sheaves. By changing the index category of sheaves one obtains different kinds of nominal sets \cite{CianciaKM10}, and different classes of HD-automata. Since \hdmas{} are based on HD-automata, this correspondence seems relevant also for our work. For instance, by taking sheaves over graphs \cite{SammartinoPhD}, one could express complex relations among symbols in the alphabet, and require that, infinitely often, one encounters a symbol which is related in a certain way to a number of its predecessors. Furthermore, recall that automata correspond to logic formulae: \hdmas{} could be used to represent logic formulae with binders; it would also be interesting to investigate the relation with first-order logic on nominal sets \cite{Bojanczyk13}. There may be different logical interpretations of \hdmas, where causality or dependence \cite{Vnnen07,Galliani12} between events are made explicit. Finally, extending the two-sorted coalgebraic representation of Muller automata introduced in \cite{CianciaV12} to \hdmas{} would yield canonical representative of automata up to language equivalence.

%setting of \cite{\cite{CianciaV12}} to hDMAs would enhance the defined framework with canonical representatives of automata, up-to language equivalence.\todo{Non è molto comprensibile}



%a subcategory of presheaves indexed by finite sets and injections \cite{fabio,staton}, namely the category of \emph{sheaves} with respect to the atomic topology (the so-called \emph{Schanuel topos}). HD-automata, that are the finite-words \hdmas, have been proven equivalent to coalgebras on sheaves \cite{Chi?}. By changing the index category of sheaves one could obtain different notions of named sets, and interesting classes of automata. For instance, one could take sheaves on graph-based network topologies \cite{matteo-tesi},

%Following this correspondence, one could take other examples 

%correspond to \emph{sheaves} with respect to the atomic topology (the so-called \emph{Schanuel topos}). These form form a subcategory of presheaves indexed by finite sets and injections \cite{fabio,staton}. 


%By changing the index category, one may express more complex relations among symbols in the alphabet, such as graph-based network topologies \cite{matteo}. Extending nDMA to work in such a richer setting seems relevant. For example, using graphs \todo{... to represent relations}, one could predicate on arbitrary relations between symbols, and require that infinitely often, one passes by a symbol which is related in a certain way to a number of its predecessors. Furthermore, recall that automata correspond to logic formulae. 

%Nominal Muller automata could be put in correspondence with logic formulae with binders. However, 
